\documentstyle{article}
\parskip 1ex
\parindent 0em
\oddsidemargin -10mm \evensidemargin -10mm
\marginparwidth 40pt \marginparsep 5mm
\topmargin -10mm
%\headheight 0pt
%\footheight 0pt
\textheight 9in
\textwidth 7in
\newcommand{\abc}{{\tt abc}}
\newcommand{\abctmtex}{{\bf ABC2M\TeX}}

\begin{document}
\title{The \abc\ Indexing Guide \\ {\it Version 1.6.1 -- January '97}}
\author{Chris Walshaw \\
{\it School of Maths, University of Greenwich, London, G.B.} \\
email: {\tt C.Walshaw@gre.ac.uk}
}
\date{}

\maketitle

\section{Introduction}

This document describes how to index your tunes and facilities for
searching the database. It has been separated from the userguide
in order to keep both documents compact and to the point. It is assumed
that the reader is familiar with the userguide before reading this document.

Those people who have used the indexing facility in versions 1.1 and previous
will notice some major changes to this part of the package.
There are no longer any restrictions on the tune-file names (although you will
find it easier if all  your tune files have the extension {\tt .abc}) and
information is no longer inferred from  these file names. In addition, the
information put into the {\tt index} is chosen by the user and not confined to
what the author thinks will be useful.

\subsection{The format file}

The essential idea behind the new formatting is that the user supplies a format
file {\tt index.fmt} specifying exactly which fields (s)he wants to appear and
the maximum number of characters that each should have. For example
the `compulsory' format in versions 1.1 and previous could still be used if
\begin{verbatim}
F<4 X>3 O<1 R<2 T:55
\end{verbatim}
were the contents of {\tt index.fmt}. The letters of this format correspond
to \abc\ fields (save {\tt F} which is the file name with the path
and the {\tt .abc} extension removed) and imply that the first
4 characters are the file name, followed by a space, followed by the tune
number (3 characters), space, origin (1 character), space, tune type (2
characters), space, title (up to 55 characters). The symbols {\tt <}, {\tt >}
and {\tt :} mean that the text of the field is left justified \& padded with
whitespace, right justified \& padded and left justified without padding,
respectively. The colon, {\tt :}, can only
be used for the last field of each line (otherwise the code won't be able to
untangle the format).

WARNING: The code looks at {\em all} the characters in the format file. If you
are using a PC or a Mac make sure that there is
a newline character at the end of every line including the last!

A few more example formats are:--
\begin{itemize}
\item The format I use now is a two line format and has in addition the meter,
default note length, key and first two bars of each tune. Its format file looks
like
\begin{verbatim}
F<4 X>3 O<1 R<2 T:58
 M>5 L>4 K<5 |:54
\end{verbatim}
\item If your tunes are stored one to a file there's no need to specify the
file name or number, but you might want the composer
\begin{verbatim}
M>5 C<18 T:55
\end{verbatim}
\item If you want to use the UNIX sort facility in addition to the one provided
here, you might want to put field delimiters, e.g. a {\tt ;}, in the index
\begin{verbatim}
F<4;X>3;O<1;R<2;T:58
\end{verbatim}
\item By default the {\tt .abc} extension is dropped from the file name. If you
want to put it back in
\begin{verbatim}
F>8.abc R<8 T:50
\end{verbatim}
\item To output a character that is meaningful to the formatting
code, for example to generate a listing that looks like an \abc\
tune file, it is possible to quote the character with a \verb+\+
\begin{verbatim}
\X:X:8
\T:T:55
\O:O:10
\R:R:10
\M:M:5
\L:L:4
\K:K:5
|:60

\end{verbatim}
\item Similarly, to include a \verb+\+ in the index, for example to embed
\TeX\ commands, use \verb+\\+
\begin{verbatim}
\\texcommand F<8.abc T:50
\end{verbatim}
\end{itemize}

\subsection{Information fields} \label{sec:info}

The following information fields can be used for indexing:--
\begin{itemize}
\item[{\tt X}] -- tune reference number.
\item[{\tt T}] -- tune title. For the purposes of indexing,
definite \& indefinite articles should be placed
at the end after a comma -- e.g. `The Tenpenny Bit' should be entered as
{\tt T:Tenpenny Bit, The}, although this will appear as `The Tenpenny Bit'
on the printed music sheet. The normal \TeX\ accents (such as in \verb+caf\'e+
for producing caf\'e) are stripped out of the index.
\item[{\tt C}] -- composer.
\item[{\tt M}] -- meter.
\item[{\tt L}] -- default note length.
\item[{\tt O}] -- origin, e.g. {\tt O:I} or {\tt Irish} -- Irish, {\tt O:E} --
English, \ldots.
\item[{\tt R}] -- tune type, e.g. {\tt R:R} or {\tt R:reel} -- reel, {\tt R:J}
or {\tt R:jig}
-- jig, \ldots.
\item[{\tt K}] -- key.
\end{itemize}

The {\tt O:} \& {\tt R:} fields can be used to create a default for
each tune file in exactly the same way as the {\tt M:} field and can be reset
later in the file.

Two further fields can be specified in {\tt index.fmt} -- {\tt F} for referring
to the file name and {\tt |} for putting in up to two bars from the beginning
of the tune. The default is two bars plus leading notes, but to change this use

\begin{tabular}{ll}
{\tt |1}  & first bar without leading notes \\
{\tt |2}  & first two bars without leading notes \\
{\tt |01}  & first bar with leading notes \\
{\tt |02}  & first two bars with leading notes
\end{tabular}

Note that, for the purposes of searching the index, there
is no point in putting in the first bars without the key, {\tt K} field (unless
all your tunes are in {\tt C}). Also grace notes will be ignored and only
the first note of a chord will be used.

\subsection{Making the Index}

To make an index of all the tunes stored in {\tt .abc} files simply enter
{\tt abc2mtex -i *.abc}. The result is stored in the file
{\tt index} which can then be sorted alphabetically by title with
the command {\tt sort\_in}. On a UNIX machine this can all be done
with the command {\tt make index}.

To do this manually or to select files without a {\tt .abc} extension or even
select specific tunes run {\tt abc2mtex -i}. If you are repeating
this operation frequently, put a list of the required files in
another file (one per line) and then run {\tt abc2mtex -i < list}.

\section{Sorting}

If you have used the {\tt make index} command the index is
automatically sorted, otherwise, enter
{\tt make sort} or just {\tt sort\_in} if this program is already compiled.

By default, {\tt sort\_in} reads in the file {\tt index} using the
corresponding format file {\tt index.fmt}, sorts it and puts
it back where it came from. To sort a different file, e.g.
{\tt jigs\_index}, run {\tt sort\_in -f jigs\_index} (which will
need to use the format file {\tt jigs\_index.fmt}). To sort the
standard input (for example to sort the output
of {\tt search}), run {\tt sort\_in -f -} (which will read the
format file {\tt index.fmt}).

Also, by default, {\tt sort\_in} sorts alphabetically by title and if
two titles are identical their positioning is indeterminate. To
change the sorted field or to specify a sorting priority, run with
a {\tt -p} flag followed by the order in which fields should be
sorted. For example, {\tt sort\_in -p TRFX}, will first compare
two entries by title. If they are identical it will then compare
tune type, then file name, then number. {\tt sort\_in -p RT} will
sort the index by tune type, then by title. The fields that can be
used by this option are those given in section~\ref{sec:info}.

Finally, when comparing character strings, {\tt sort\_in} only
looks at A-Z, a-z, 0-9 and whitespace characters, and folds
upper case down into lower case.

\section{Searching}

Ever since development first started on \abctmtex\ I have been
planning a facility that enables the user to search through tune
files to look for a tune beginning with a certain sequence of notes.
The code that has been implemented will do not only this but
search any of the specified fields in the {\tt index} or tune files.

\subsection{What to do}

First make the {\tt index} as described above and compile the search code;
{\tt make search}.

The search is started by entering the command {\tt search}, followed
by selected fields, e.g.
\begin{verbatim}
%search
M:6/8
K:D
dff cee|def gfe|

\end{verbatim}
or
\begin{verbatim}
%search
O:Irish
R:jig

\end{verbatim}
Note that the final blank line tells the code to start the search.

Alternatively the search input can be put on the command line, e.g.
\begin{verbatim}
%search -i M:6/8 -i K:D -i "dff cee|def gfe|"
\end{verbatim}
or
\begin{verbatim}
%search -i O:Irish -i R:jig
\end{verbatim}
Note the use of double quotes to enclose input with whitespace.

As in the tune files, if you wish to search through the first bars of each
tune, the {\tt K:} field should be the last and
should immediately precede the notes. Also, chords \& leading
notes should be left out and grace notes will be ignored.

\subsection{Searching the {\tt index}}

As for {\tt sort\_in}, by default, {\tt search} reads in the file {\tt index}
using the corresponding format file {\tt index.fmt}. To search a different
file, e.g.  {\tt jigs\_index}, run {\tt search -f jigs\_index} (which will
need to use the format file {\tt jigs\_index.fmt}).

The information fields can be used to restrict the search to particular parts
of the index. Fields recognised by the code are the same as those allowed by
the indexing under the following conditions:--
\begin{itemize}
\item[{\tt M:}] -- meter; if you use this field the code will only find
tunes in this meter. If you leave it out the default note length
(for the \abc\ input) is a quaver (1/8th note) although this can be reset with
the {\tt L:} field. You can also use {\tt M:2} to search all meters divisible
by two but not by three (e.g. {\tt M:2/2}, {\tt M:2/4},
{\tt M:4/4}, {\tt M:C}, {\tt M:C|}) or {\tt M:3} to search all
meters divisible by three (e.g. {\tt 3/2}, {\tt 3/4}, {\tt 6/4}, {\tt 3/8},
{\tt 6/8}, {\tt 9/8} \& {\tt 12/8}).
\item[{\tt L:}] -- this can be used to set the default note length
for the \abc\ input -- it will {\em not} restrict the search to tunes with the
same default note length.
\item[{\tt K:}] -- key; this must be used to notify the code that the following
line is the \abc\ input. However the search is not restricted to any
single key.
\item[{\tt T:}] -- title; to search for tunes by name. Note that
this restricts the search to any tune with the entered string
in the name. Capitals are ignored and the `.' can be used as a wildcard
to match any letter. In addition, this field can be used more
than once to look for more than one string. Thus the input
\begin{verbatim}
T:b.rwick
T:jo...y

\end{verbatim}
would match the following strings:--
\begin{verbatim}
Barwick Johnny
Go To Berwick Johnny
Go To Berwick My Jockey
\end{verbatim}
\item[{\tt C:}] -- composer; restricts the search to any tune with the entered
string in the composer field.
\item[{\tt O:}] -- origin; restricts the search to exact matches of this field.
\item[{\tt R:}] -- tune type; restricts the search to exact matches.
\item[{\tt X:}] -- tune number; restricts the search to exact matches.
\item[{\tt F:}] -- file name; restricts the search to exact matches.
\end{itemize}

There is obviously no point entering a search field that doesn't exist in
{\tt index.fmt} and hence the {\tt index}. No matching tunes will be found.

When searching for notes, the code compares the input notes with
up to two bars (ignoring leading notes) of each entry in the {\tt index} (those
not otherwise excluded by other fields) and calculates the `distance'
between them -- as described below. If the distance is less than
a certain level it then outputs the index entry. By default this level
is set to 6, but if you find this either too restrictive
or too generous you can alter it by calling the program with the chosen
level as a command line input; e.g. enter {\tt search 8} to expand the
search.

\subsection{Searching the tune files}

Alternatively, {\tt search} can scan the tune files directly with the command
{\tt search -abc} followed by a list of the tune files to read
(e.g. {\tt search -abc *.abc}). In this case there is no restriction
on the fields that can be searched, however, the way code
matches the input is subtly different. The {\tt M:}, {\tt L:} \&
{\tt K:} fields all work as described above. The {\tt X:} field
can be given a range of numbers -- e.g. {\tt X:1,3-4,7-} -- exactly
the same as when selecting tunes for \abc2mtex. Any other fields
behave the same way as the {\tt T:} field above. That is, the code
tries to match the string entered with any matching field. For
example, searching for {\tt R:jig} would match {\tt R:Jig}, {\tt R:Slip jig}
or even {\tt R:SINGLE JIG}.

The output of this type of search is standard \abc\ code (with the addition of
an {\tt F:} field which specifies the filename) and can be piped directly into
\abc2mtex. For example, {\tt search -abc *.abc | abc2mtex -i -}, would create an
index of the matching tunes. Alternatively, it could be used as a
preprocessor to select tunes for {\tt playabc}, i.e.
{\tt search -i X:1,6,9-11 -abc Xenglish.abc | playabc | tune | play}.

\subsection{How does the note matching work?}

The code first rewrites the input string as a sequence of demi-semi-quavers
(1/32th notes), ignoring grace notes, rolls, etc. and dropping the second note
of triplets and quadruplets. Thus the input of
\begin{verbatim}
L:1/8
K:G
GFG (4ABcA B2G|
\end{verbatim}
would be written as
{\tt G G G G F F F F G G G G A A A A c c c c A A A A B B B B B B B B G G G G}.
% Next each pair of notes in this sequence is considered and the change between
% them in semitones calculated; e.g.
% \begin{verbatim}
% G G  F F G G A A c c  A A B B B B  G G
%  0 -1 0 1 0 2 0 3 0 -3 0 2 0 0 0 -4 0
% \end{verbatim}
Next each note in this sequence is considered and the change in semitones
between it and the first note is calculated; e.g.
\begin{verbatim}
G G G G  F  F  F  F G G G G A A A A c c c c A A A A B B B B B B B B G G G G
  0 0 0 -1 -1 -1 -1 0 0 0 0 2 2 2 2 5 5 5 5 2 2 2 2 4 4 4 4 4 4 4 4 0 0 0 0
\end{verbatim}
Finally two sequences of numbers are compared by taking the absolute value
of the difference between each corresponding pair of numbers.

\subsubsection{What it can do}

Because notes are compared relative to the
first
% previous
one and the difference expressed in semitones the code doesn't care about keys
at all. Thus {\tt GFGB} in {\tt K:G} is identical to \verb+d^c d^f+ in
{\tt K:C}. Also because everything is expressed as demi-semi-quavers
(1/32th notes)
prior to matching {\tt AB c2} in {\tt L:1/8} is the same as {\tt A2BB c2 cc}
in {\tt L:1/16}.

\subsubsection{What it can't do}

Again, because everything is expressed as demi-semi-quavers, the line
{\tt defd} in {\tt M:4/4} does not compare to the same line in {\tt M:2/4}.
This might cause a problem if, for example, you have some reels written in
{\tt M:C|} and some in {\tt M:2/4}. The only solution is do the search twice,
once with {\tt L:1/8} and once with {\tt L:1/16}.

\end{document}

